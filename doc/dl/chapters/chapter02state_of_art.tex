\textit{Yuan et al. (2012)} \cite{AIReview}, in their trajectory clustering review, show some techniques widely used for moving object clustering. The latest Machine Learning algorithms are all evolution and improvement of the traditional clustering techniques: K-Means, Mean Shift, Gaussian Mixture, Hierarchy, Agglomerative, etc. For spacial based clustering, \textit{Palma et al. (2008)} \cite{Palma2008} considers trajectories geometrical properties to detect pauses and moves in trajectories by using an improved version of \textit{DBSCAN} algorithm. For time depended clustering, \textit{Nanni and Pedreschi (2006)} \cite{Nanni2006} considers time gaps between trajectory positions and proposes \textit{T-OPTICS} as an adaption of \textit{OPTICS}. For partition based clustering, \textit{Lee et al. (2007)} \cite{Lee2007} implement a \textit{partition \& group} framework for trajectory clustering based on a formal theory. For uncertain trajectory clustering, Nock and Nielsen 2006 \cite{Nock2006} implemented \textit{Fuzzy C-Means (FCM)} that model the uncertainty and reduce the noise of trajectories location between discrete time updates. For semantic trajectory clustering, several works start considering geographical informations as a background from which extracting higher semantic level information (\textit{Palma et al. (2008)} \cite{Palma2008} stops and moves).

On the other hand, the main Deep Clustering techniques are based on autoencoders and was first introduced with the \textit{D. Yao et al. (2017)} \cite{DeepLearningRepresentation} work, in which they present a technique based on a sliding window that extract behavior feature along the trajectories and further employ a sequence to sequence autoencoder to learn fixed-length deep representation. The autoencoder based clustering, introduced in \textit{D. Yao et al. (2017)} \cite{DeepLearningRepresentation} work, reduces the trajectory representation in a latent space contained in the neural network architecture, but then a traditional clustering algorithm has to be applied in order to locate similar trajectories. \textit{Olive et al. (2020)} \cite{DeepLearningAutoencoders} employs a deep clustering method named \textit{DCEC (Deep Convolutional Embedded Clustering)}, that uses a convolutional autoencoder and it can identify the clusters directly without the need of using a classical clustering method. In the same work (\cite{DeepLearningAutoencoders}) they implement \textit{artefact}, a method based on a feed-forward autoencoder and a loss function combining the reconstruction error and a regularisation term based on the projection operator \textit{t-SNE}. \textit{M. Yue et al. (2019)} \cite{DETECT} have currently produced the best deep clustering solution that is named \textit{DETECT (Deep Embedded TrajEctory ClusTering network)}. \textit{DETECT} operates in three parts: first it transforms the trajectories by selecting its own point-of-interest for each position based on its context derived from their geographical locality, then they use autoencoder to learn trajectory representation and finally the a clustering oriented loss is directly built on the latent space features to refine the cluster assignment.

Furthermore, there are works on Deep Clustering that do not adopt the autoencoder architecture. \textit{Mahdi M. K. et al. (2015)} \cite{MotionPatternApproach} proposed an algorithm composed by four phases in which they break down trajectories into flow vectors that indicate instantaneous movements and then they apply different clustering steps in order to extract similarity with respect to their location, velocity, spacial proximity, motion and reachability. \textit{Manduchi L. (2020)} \cite{DPSOM} present \textit{DPSOM (Deep Probabilistic Clustering with Self-Organizing Maps)}, that is based on the \textit{SOM (self-organizing-map)}, which is a clustering technique that creates a nuance between the boundaries of the cluster, thus giving a more flexible interpretation of the clusters. In particular the \textit{T-DPSOM} algorithm outperforms baseline methods in time series clustering and time series forecasting.

In this report I will present some partition and group heuristics, based on time gaps, I will show how the traditional clustering algorithms can be used for trajectory clustering relying on positions spatial information and I will compare the results with Deep Clustering techniques based on the autoencoder architecture explained in \cite{DeepLearningRepresentation}. 
